\documentclass[a4paper,12pt]{article}
\usepackage[utf8x]{inputenc}
\usepackage{graphicx}
\usepackage[margin=1in]{geometry}
\usepackage{subfig}
\usepackage[countmax]{subfloat}
\newcommand{\tab}{\hspace*{2em}}
%opening

\begin{document}


\begin{center}
{\Large Efficiently Generating Code for the GPU}

\emph{Matthew Le}
\end{center}

I am currently working on two research projects, both of which are funded by the NSF Research Experience for Undergraduates program. The first project is under the guidance of Professor Eric Van Wyk, dealing with finding ways to design programming languages, such that they can easily be extended. The second project that I have been working on deals with investigating the effects of Pacific Ocean warming on Atlantic hurricane activity. For this project, I have been working closely with a graduate student, his advisor, and another research scientist.  

Domain specific languages have become popular because they allow the programmer to write code at a high level of abstraction, while enjoying the benefits of domain specific optimizations. The downside of creating a new language to add these domain specific features is that you lose portability. Also, since it is a different language, previously existing code must be rewritten in the new language. Our research aims to find ways to implement these domain specific features as language extensions to already existing general purpose languages. This way, previously written code can still be utilized, and the extensions can be used on all platforms that the host language is built for.  

The project that I am working on implements this idea by first creating a host language. For this, we have chosen a prototype language, which is a subset of C. The role of the host language is to show the state of a general purpose programming language in its initial state without any extensions.  We chose to use a subset of C because its low level of abstraction gives rise to programming complexity, but when used correctly it can yield code that runs very efficiently. After the host language was implemented, we began working on extensions. The extension that I have been working on adds support for matrices, and the operations that accompany them. We create these extensions using a technique known as ``forwarding'', which works by reducing extensions to constructs that already exist in the host language.  The extension that I am working on makes heavy use of forwarding, such that this extensions is ``pure'', meaning it is completely reducible to constructs existing in the host language.  The benefit to this is that if the host language gets updated, then the extension does not need to be modified.  Also, this allows for other language extensions to be simultaneously and independently implemented.

I began implementing the extension by first designing the underlying representation of matrices. Since we want this matrix extension to generalize to any number of dimensions, we chose to represent matrices as a structure containing the size of each dimension, and the data the matrix contains. We then index matrices using row-major indexing. After designing the underlying representation, we created an additional type for matrices, which then forwards to a host type, in this case we forward to the host struct type. We then created syntax and abstract productions for declaring matrices.  Matrix declarations are really no different than declaring any other variable, we simply need to add the name of the matrix being declared to the symbol table. Another necessity that we implemented was matrix initialization. This production forwards to a sequence of assignment statements, where we allocate space for the structure, the dimension sizes array, and the data array. A few other productions that we added to the core matrix extension include matrix references, matrix arithmetic, and printing of matrices.  

After creating an extension that simply introduces this new data structure and some very basic operations, we started adding some of the domain specific features that are found in MATLAB.  Such additional functions include logical indexing, referencing multiple elements of a matrix at a time, assigning to multiple elements of a matrix, and matrix permutations.  We ended up putting this functionality into an extension of its own, showing that not only host languages can be extended in a modular way, but extensions can also be further extended in a modular way.  These additional features have presented many opportunities for generating parallel C code, which is something that we are currently exploring now.

As of now, we currently have a functioning language extensions that supports just about everything that one would be able to do in MATLAB. We are able to generate C code from this extension, and are currently working on the optimization phase of the project. I have implemented many of the algorithms that we use in the data mining research lab that I am also apart of, using the language extension that I have created. The two versions of most algorithms run in about the same amount of time, and produce the same results. The fact that these two versions run in a similar amounts of time, without any optimizations is very encouraging, because this implies that once we are able to generate parallel code, we will have a large runtime advantage over MATLAB on top of the fact that we have successfully integrated this extension into a general purpose language.

A second research opportunity that I am currently working on is another NSF REU funded project in the field of climate data mining, under the guidance of graduate student James Faghmous and his advisor Professor Vipin Kumar.  The El-Ni\~{n}o Southern Oscillation (ENSO) is a phenomenon where Pacific Ocean warming has been linked to Atlantic hurricane activity\cite{enso}.  The problem is that the correlation between the two is quite poor.  This research effort aims to further investigate this relationship. 

When I began working on this project, James had already developed a method that explained the Pacific-Atlantic relationship more accurately than ENSO.  The idea behind it is that instead of monitoring the temperature of the Pacific Ocean, you monitor the location of the warming, i.e. what subset of the Pacific is the warmest.  This index correlated with hurricane counts in the Atlantic quite a bit better than the ENSO index.  However, Professor Kumar seemed reluctant to move forward with this method.  In the meantime, I had been experimenting with some variations of this index.  Many of these variations involved incorporating additional variables in addition to sea surface temperature.  After a few months, I had come up with an index of my own that performed a fair amount better than the previous one.  I had found that if I were to apply this spatial monitoring idea to different variables, I could come up with multiple separate indices, and then by taking their z-score and summing them together, we had one index that encapsulated the information of a few different variables.  This index has remained our top performing predictor of Atlantic hurricane activity when we restrict ourselves to Pacific Ocean information.  Over the past few months we have been compiling our results and testing the validity of this index.  Professor Kumar anticipates that it won't be long until we are able to publish these results.

In addition to conducting research, I have also had a few experiences with presenting my results.  About four months into working on my programming languages research project, we presented at the University of Minnesota Undergraduate Research Symposium.  this was an especially interesting experience because we were not just presenting to computer scientists.  It proved to be a rather difficult task to explain what we were working on to people who had never been exposed to computer science before. I also had the opportunity to present some of our work in the data mining research lab at the Expeditions in Computing Workshop hosted by the University of Minnesota. 

I have had a great time working on these two projects, and feel that they have stimulated my interest in conducting research in the future.  I certainly have a different perspective on conducting research now that I have had some exposure to it, and feel that I am much more prepared than previously in carrying out research at the graduate level.  
\begin{thebibliography}{9} \vspace{-1ex}
\footnotesize
\bibitem{enso}M.C. Bove, J.J. O'Brien, J.B. Elsner, C.W. Landsea, X. Niu.  Effect of El Nino on U.S. Landfalling Hurricanes, Revisted.  \textit{Bulletin of the American Meteorological Society}, Volume 79, Number 11
\end{thebibliography}
\end{document}










