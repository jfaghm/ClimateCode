\section{Introduction}



The quasi-periodic cycle of warming and cooling of the near equatorial Pacific sea surface temperatures (SST), known as the El-Ni\~no Southern Oscillation (ENSO), has well-documented global teleconnections. A common pathway by which Pacific SST warming affects the globe is through the alteration of the Walker circulation due to ---. 

For the last 50 years, scientists have attempted to abstract such a cycle using empirical warming-based indices. Indices such as NINO1+2 and NINO3.4 are constructed by averaging the sea surface temperature (SST) anomalies of static oceanic regions and are further correlated to land and ocean phenomena. While such indices have been a staple of teleconnection research, an increasing number of studies report a shifting in the spatial warming patterns of the Pacific therefore making the monitoring of fixed regions less informative (see Figure 1). 

We propose that instead of monitoring the intensity of warming within static regions, we dynamically monitor the location of the warmest SST region in the near equatorial Pacific. We propose a distance-based ENSO index that tracks the longitudinal location of highest SST anomaly in the tropical Pacific. Such an index has the advantage of capturing both the intensity and location of ENSO.



